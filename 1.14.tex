\documentclass{article}
\usepackage{tikz}
\usetikzlibrary{arrows}
\usetikzlibrary{positioning}
\tikzset{
	treenode/.style = {align=center, inner sep=2pt, text centered, rectangle, black, font=\sffamily\bfseries, draw=black, fill=white},
	endright/.style = {draw=red},
	endleft/.style  = {draw=blue},
}

\begin{document}
The orders of growth for the count-change program for the number of steps is $ \Theta(n^5) $ because for every time the kinds-of-coins variable is be incremented, there will be approximately just as many new steps for every step that already existed. So if the count-change program had consisted of only three types of change, the order of growth would have been $ \Theta(n^3) $. $ \Theta(n^5) $ means that $ \Omega(n^5) $ and also $ \mathcal{O}(n^5) $. The space required for the program is $ \Theta(n) $ because the program only needs to keep track of the nodes above the current nodes, so in the end (count-change amount 1) will compose the longest part of the tree.

\hspace*{-4em}
\begin{tikzpicture}[
	level 1/.style={sibling distance = 3cm},
	level 3/.style={sibling distance = 9cm},
	level 4/.style={sibling distance = 2cm},
	->,
	>=stealth',
]
\node {cc 11 5}
	child{ node {cc 11 4} 
			child{ node {cc 11 3} 
				child{ node {cc 11 2} 
					child{ node (a10) [left = 2cm] {cc 11 1}
						child{ node [left=of a10][endleft] {cc 11 0}}
						child{ node (a9) [below=of a10] {cc 10 1}
							child{ node [left=of a9][endleft] {cc 10 0}}
							child{ node (a8) [below=of a9] {cc 9 1}
								child{ node [left=of a8][endleft] {cc 9 0}}
								child{ node (a7) [below=of a8] {cc 8 1}
									child{ node [left=of a7][endleft] {cc 8 0}}
									child{ node (a6) [below=of a7] {cc 7 1}
										child{ node [left=of a6][endleft] {cc 7 0}}
										child{ node (a5) [below=of a6] {cc 6 1}
											child{ node [left=of a5][endleft] {cc 6 0}}
											child{ node (a4) [below=of a5] {cc 5 1}
												child{ node [left=of a4][endleft] {cc 5 0}}
												child{ node (a3) [below=of a4] {cc 4 1}
													child{ node [left=of a3][endleft] {cc 4 0}}
													child{ node (a2) [below=of a3] {cc 3 1}
														child{ node [left=of a2][endleft] {cc 3 0}}
														child{ node (a1) [below=of a2] {cc 2 1}
															child{ node [left=of a1][endleft] {cc 2 0}}
															child{ node [below=of a1] {cc 1 1}
																child{ node [endleft] {cc 1 0}}
																child{ node [endright] {cc 0 1}}
															}
														}
													}
												}
											}
										}
									}
								}
							}
						}
					}
					child{ node {cc 6 2}
						child{ node (b5) {cc 6 1}
							child{ node [left=of b5][endleft] {cc 6 0}}
							child{ node (b4) [below=of b5] {cc 5 1}
								child{ node [left=of b4] [endleft] {cc 5 0}}
								child{ node (b3) [below=of b4] {cc 4 1}
									child{ node [left=of b3] [endleft] {cc 4 0}}
									child{ node (b2) [below=of b3] {cc 3 1}
										child{ node [left=of b2] [endleft] {cc 3 0}}
										child{ node (b1) [below=of b2] {cc 2 1}
											child{ node [left=of b1] [endleft] {cc 2 0}}
											child{ node [below=of b1] {cc 1 1}
												child{ node [endleft] {cc 1 0}}
												child{ node [endright] {cc 0 1}}
											}
										}
									}
								}
							}
						}
						child{ node [right = 1cm] {cc 1 2}
							child{ node {cc 1 1}
								child{ node [endleft] {cc 1 0}}
								child{ node [endright] {cc 0 1}}
							}
							child{ node [endright] {cc -4 2}}
						}
					}
				}
				child{ node {cc 1 3} 
					child{ node {cc 1 2}
						child{ node {cc 1 1}
							child{ node [endleft] {cc 1 0}}
							child{ node [endright] {cc 0 1}}
						}
						child{ node [endright] {cc -4 2}}
					}
					child{ node [endright] {cc -9 3}}
				}
			}
			child{ node [endright] {cc -14 4}}
	}
	child{ node [endright] {cc -39 5}}
; 
\end{tikzpicture}
\end{document}
